%%%%%%%%%%%%%%%%%%%%%%%%%%%%%%%%%%%%%%%%%%%%%%%%%%%%%%%%%%%%%%%%%%%%%%
% writeLaTeX Example: A quick guide to LaTeX
%
% Source: Dave Richeson (divisbyzero.com), Dickinson College
% 
% A one-size-fits-all LaTeX cheat sheet. Kept to two pages, so it 
% can be printed (double-sided) on one piece of paper
% 
% Feel free to distribute this example, but please keep the referral
% to divisbyzero.com
% 
%%%%%%%%%%%%%%%%%%%%%%%%%%%%%%%%%%%%%%%%%%%%%%%%%%%%%%%%%%%%%%%%%%%%%%
% How to use writeLaTeX: 
%
% You edit the source code here on the left, and the preview on the
% right shows you the result within a few seconds.
%
% Bookmark this page and share the URL with your co-authors. They can
% edit at the same time!
%
% You can upload figures, bibliographies, custom classes and
% styles using the files menu.
%
% If you're new to LaTeX, the wikibook is a great place to start:
% http://en.wikibooks.org/wiki/LaTeX
%
%%%%%%%%%%%%%%%%%%%%%%%%%%%%%%%%%%%%%%%%%%%%%%%%%%%%%%%%%%%%%%%%%%%%%%

\documentclass[10pt,landscape]{article}
\usepackage{amssymb,amsmath,amsthm,amsfonts}
\usepackage{listings}
\usepackage{multicol,multirow}
\usepackage{calc}
\usepackage{ifthen}
\usepackage[landscape]{geometry}
\usepackage[colorlinks=true,citecolor=blue,linkcolor=blue]{hyperref}

\lstset{
basicstyle=\ttfamily,
frame=single
}


\ifthenelse{\lengthtest { \paperwidth = 11in}}
    { \geometry{top=.5in,left=.5in,right=.5in,bottom=.5in} }
	{\ifthenelse{ \lengthtest{ \paperwidth = 297mm}}
		{\geometry{top=1cm,left=1cm,right=1cm,bottom=1cm} }
		{\geometry{top=1cm,left=1cm,right=1cm,bottom=1cm} }
	}
\pagestyle{empty}
\makeatletter
\renewcommand{\section}{\@startsection{section}{1}{0mm}%
                                {-1ex plus -.5ex minus -.2ex}%
                                {0.5ex plus .2ex}%x
                                {\normalfont\large\bfseries}}
\renewcommand{\subsection}{\@startsection{subsection}{2}{0mm}%
                                {-1explus -.5ex minus -.2ex}%
                                {0.5ex plus .2ex}%
                                {\normalfont\normalsize\bfseries}}
\renewcommand{\subsubsection}{\@startsection{subsubsection}{3}{0mm}%
                                {-1ex plus -.5ex minus -.2ex}%
                                {1ex plus .2ex}%
                                {\normalfont\small\bfseries}}
\makeatother
\setcounter{secnumdepth}{0}
\setlength{\parindent}{0pt}
\setlength{\parskip}{0pt plus 0.5ex}
% -----------------------------------------------------------------------

\title{Quick Guide to LaTeX}

\begin{document}

\raggedright
\footnotesize

\begin{center}
     \Large{\textbf{Pretty Jupyter Cheat Sheet}} \\
\end{center}
\begin{multicols}{3}
\setlength{\premulticols}{1pt}
\setlength{\postmulticols}{1pt}
\setlength{\multicolsep}{1pt}
\setlength{\columnsep}{2pt}

\section{Notebook-Level Metadata}

\textbf{YAML header method} First cell of type \textbf{raw}.
\begin{lstlisting}
title: My title
author: My name
date: "2022-01-01"
output:
    general:
        input_jinja: false
        input: true
        output_error: false
    html:
        toc: true
        toc_depth: 3
        number_sections: false
        code_folding: hide
        code_tools: false
        theme: bootstrap

\end{lstlisting}

\section{Cell-Level Metadata}
\textbf{Code Cell}
\begin{lstlisting}
# -.-|m { input: true, output: true }

# now we can continue as normal
a = 10

\end{lstlisting}

\textbf{Jinja Markdown Cell}
\begin{lstlisting}
%%jmd
[//]: # (-.-|m { input: true, output: true })

Jmd cell with visible input.
\end{lstlisting}

\section{Jinja Markdown}
\begin{lstlisting}
a = 10
b = 5
\end{lstlisting}

\begin{lstlisting}
%%jmd

We can combine variables like
this: {{ (a / b) | round(2) }}.
\end{lstlisting}

\begin{lstlisting}
We can also use advanced Jinja
syntax like this:


{{ a }}

{{ b }}

\end{lstlisting}

\section{Table of Contents}

\begin{lstlisting}
%%jmd
# Chapter 1

## Section 1

## Ignored Section 2
[//]: # (-.- .unlisted .unnumbered)

## Section 3

This section will not appear in TOC.
\end{lstlisting} 

\section{Tabset}
\begin{lstlisting}
%%jmd

# Tabset Root
[//]: # (-.- .tabset .tabset-pills)

## Tab 1

## Tab 2
\end{lstlisting}

\section{Code Folding}
\begin{lstlisting}
# -.-|m { input_fold: show }
a = 10
\end{lstlisting}

\section{Styling}
\begin{lstlisting}
%%jmd

| col1 | col2 |
|------|------|
| val1 | val2 |

[//]: # (-.- #table-id)

<style>
    #table-id {
        background-color: red;
    }
</style>
\end{lstlisting}

\begin{lstlisting}
%%jmd

Alert text.

[//]: # (-.- .alert .alert-warning)
\end{lstlisting}

\section{CLI}

\begin{lstlisting}
pretty-jupyter quickstart /path/to/ipynb
\end{lstlisting}

\begin{lstlisting}
jupyter nbconvert --to html --template pj
/path/to/ipynb/file
\end{lstlisting}



\end{multicols}

\end{document}
